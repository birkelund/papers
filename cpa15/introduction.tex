\section*{Introduction}
Research on large-scale storage systems in relation to energy consumption is
an increasingly active area with many recent developments\cite{survey-power}. This is
motivated by the fact that storage needs are increasing rapidly and
exponentially in the data center and thus the energy bill associated
with storage becomes a relatively large part of running an I/O intensive
business.

Much of this research has focused on methods and techniques for optimizing
existing storage and file systems; especially in the area of power-aware
RAID\cite{paraid} or similar energy-efficient and durable storage. To this end,
we are developing a system that allows us to simulate, prototype and validate
the implementation of a storage system.  That system is not the topic of this
paper, but serves as an important case study and context in which we will
present the concurrent architecture (or \emph{design pattern}), called
Interchangeable Simulation and Implementation (ISI), that we have identified
while developing the simulator. Our system should be a power-aware simulator at
large-scale and provide us with designs for storage hierarchies, but we wanted
to decrease the time between simulation (\emph{design}) and prototyping
(\emph{implementation}) of the designs. Similarly we wanted to allow a form of
validation through measurements of the model.

First, this paper presents some background and our motivation is described in
context of related work on systems and storage simulation. Next, the
architecture is described followed by a description of how ISI can be used to
effectively develop highly concurrent systems that can be simulated, prototyped
\emph{and} validated using a single toolbox.
