\section{Conclusions and Future Work}
In this paper we have outlined ISI, a design for Interchangeable Simulation and
Implementation. We have argued how this concurrent architecture allows us to
minimize the work related to transitioning from a simulated system to a prototyped
implementation by modeling the system as communicating sequential processes.
This approach allows the programmer to model the functional parts of a
simulation as he or she would if directly writing an implementation. The
process-based programming model allows us to both simulate in discrete time and
measure in real time, using the same code, by replacing simulated actions with
real actions. As is true for any process-based architecture, the system
components can be easily composed and changed internally without affecting
other parts, which is an advantage if the development of a system is of an
exploratory nature where the final composition depends on simulated results and
analysis hereof.

For a process-based architecture to be usable in practice, it must perform and
scale well. Though our benchmark is based on a simplistic example, it does
show that even for at simulation completely dominated by messaging (and
thereby, large amounts of internal synchronization and memory pressure),
parallelization of the architecture is most certainly achievable. In our best
result, we achieve a 13x speed-up with two cores and buffered channels,
compared to the unbuffered single core baseline (5x for the buffered baseline)
with 250,000 processes. And in process counts, we are aiming for the millions.

Looking forward, the next step for ISI is to package the idioms and reusable
components as a Go package. Similarly, we are continuing the development of our
storage simulation system which we expect will result in further refinement and
development of the ISI pattern. While we have chosen Go for this work, any
language with light-weight process support and fast channel based messaging,
should be able to benefit from this architecture. Specifically for our work in
Go, it would be prudent to examine how to effectively manage locality for the
goroutines as the random core placement and core migration can severely limit
the achievable performance for memory-bound applications.
